\documentclass{article}
\title{Prvi skok v LaTeX}
\author{Matic Bončina}
\date{8. 11. 2023}
\usepackage{amsmath}
\usepackage{graphicx}
%to je konfiguracijsko območje
%v zavitih oklepajih so opcije (options) komand
\begin{document}
\maketitle
\pagenumbering{gobble}
\newpage

\tableofcontents
\listoffigures
\listoftables
\newpage

\pagenumbering{arabic}
\section{Uvod}
To je uvod v LaTeX, ki ga bomo uporabljali za namene študijskega izvajanja.

\subsection{Nabor hipotez}
\paragraph{Hipoteza:} Danes je lepo vreme
\subparagraph{Pod hipoteza:} Vsak dan je lepo vreme

\section{Gnezdenje}
\subsection{Prvi}
\subsection{Drugi}
\subsubsection{Tretji}

\section{Matematični izrazi}
\subsection{Oštevilčenje enačb}
\begin{equation}
    f(x) =  x^2
\end{equation}

\subsection{Neoštevilčeni izrazi}
\begin{equation*}
    g(x) = k*x + n
\end{equation*}

\begin{align*}
    1+2+3 &= 6x\\
    y &= 3x + 6\\ %"&" naredi poravnavo na ta znak
    f(x) &= \frac{x^2 + 3x + 8}{7x + 3}\\ % "\\" naredi novo vrstico
    g(x) &= \int^a_b \frac{1}{x}    
\end{align*}

\subsection{Funkcija v besedilu}
To je funkcija v besedilu: $g(x) = \int^a_b
\frac{1}{x}$.

\subsection{Matrike}
$ 
%"$" pove matematični izraz
\left[
\begin{matrix}
    1 & 1 & 0 & 1\\
    1 & 5 & 2 & 2
\end{matrix}
\right]
$

\newpage
\section{Slike}
Riba Nemo (glej slika \ref{fig:nemo}) je ogrožena.
\begin{figure}[h] % here top bottom [h][t][b] ("!" overwrite-a)
    \centering
    \includegraphics[width=\linewidth]{slike/nemo.jpg}
    \caption{Nemo k peca ribetino}
    \label{fig:nemo}
\end{figure}

\section{Tabele}
V tabeli \ref{tab:seznam} najdete kul folk.
\begin{table}[h]
    \centering
    \begin{tabular}{c c|c}
        \hline
         \textbf{Ime} & \textbf{Priimek} & \textbf{Kabinet}\\
         \hline \hline
         Matic & Bončina & Ljubljanska ulica 12 \\
         Dedi & Krompir & Ljubljanska ulica 12
    \end{tabular}
    \caption{Seznam predavateljev}
    \label{tab:seznam}
\end{table}

\end{document}
